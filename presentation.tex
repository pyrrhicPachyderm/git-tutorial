\usetheme{Madrid}
\usecolortheme{seahorse}


%A command for unnumbered footnotes.
%From https://latex.org/forum/viewtopic.php?t=30813
\newcommand\blfootnote[1]{%
	\begingroup
	\renewcommand\thefootnote{}\footnote{#1}%
	\addtocounter{footnote}{-1}%
	\endgroup
}

\title{An Introduction to Git}
\author{Christopher Brown}
\date{}

\begin{document}

\begin{frame}
	\titlepage
\end{frame}
\note[itemize]{
	\item
	I'm going to teach this over several sessions.
	Hopefully the first session will teach enough to just spin up a Git repository and start making commits.
	But if that's all you ever do, you're missing out on most of the benefits that version control will give you.
	So subsequent sessions will focus on how to get the most out of Git, and use it to be more productive.
}

\part{Prerequisites}

\begin{frame}
	\partpage
\end{frame}
\note[itemize]{
	\item
	Before we start looking at Git, there're some prerequisites we need to go over.
	
	\item
	Primarily, you'll need to know the very basics of the shell.
	We'll use Git from the command line (the shell), so it'll help to be familiar with it.
	
	\item
	There're also a few other things I'll explain the details of now, to save having to go off on tangents later.
}

\section{The Shell}

\subsection{Introduction}

\begin{frame}{Which Shell?}
	\begin{itemize}
		\item \texttt{sh}: the Bourne Shell by Stephen Bourne, 1977
		\item \texttt{bash}: the Bourne Again SHell, 1989
		\item \texttt{zsh}, \texttt{ksh}, \texttt{tcsh}, \texttt{rc}\dots
	\end{itemize}
	\texttt{bash} is default on most Linux distros.
	
	Mac OS X defaults to \texttt{zsh}.
	
	Git on Windows comes with a \texttt{bash} emulator.
\end{frame}
\note[itemize]{
	\item
	You might have heard of the shell called the ``terminal'' or the ``command line''.
	It's a text-based interface to your computer.
	You can use it instead of a graphical user interface (GUI).
	
	\item
	There are GUIs for Git, but it's good to know how to use it through the command line.
	
	\item
	Different shells, \texttt{bash}, \texttt{zsh}, etc.\ are mostly based on the Bourne shell.
	There are differences, but they're all similar enough that we shouldn't run into the differences here.
	
	\item
	On Linux or Mac OS X, you should be able to search for and open the ``terminal''.
	On Windows, you're looking for ``Git Bash''.
	Open it now.
}

\subsection{Navigation}

\begin{frame}{UNIX Directories}
	\begin{itemize}
		\item \texttt{/} is the root directory
		\item \texttt{cd \textasciitilde} is your user's home directory
		\begin{itemize}
			\item \texttt{/home/abc123/} on Linux
			\item \texttt{/Users/abc123/} on Mac OS X
			\item \texttt{/c/Users/abc123/} in Git Bash on Windows
		\end{itemize}
		\item \texttt{.} is the current directory
		\item \texttt{..} is the parent directory
	\end{itemize}
	Paths starting with \texttt{/} or \texttt{cd \textasciitilde} are \emph{absolute paths}.
	
	Others are \emph{relative paths}, and start from the current directory.
\end{frame}
\note[itemize]{
	\item
	The first thing to learn is how the directory structure is represented on the command line.
	Directories and files have \emph{paths}: a list, one inside the other, of the directories in the tree structure reaching down to them.
	Paths are separated by forward slashes.
	Windows sometimes uses backslashes, but Git Bash does it properly with forward slashes.
	
	\item
	The difference between absolute and relative paths is important in general.
	Generally, your code should use relative paths whenever possible.
	This is especially important in Git repositories.
	Suppose someone else downloads your repo: they'll have a different username, or put it in a different place, so the absolute path will be different.
	Use relative paths instead.
}

\begin{frame}{Moving Around}
	\begin{itemize}
		\item \texttt{pwd}: Print Working Directory
		\item \texttt{cd}: Change Directory
		\begin{itemize}
			\item \texttt{cd -} goes to previous directory
		\end{itemize}
		\item Right click in file manager to open command line there
	\end{itemize}
\end{frame}
\note[itemize]{
	\item
	At any point, your shell is in one directory: the \emph{working directory}.
	\texttt{pwd} (print working directory) will tell you which directory you're in.
	Try it now.
	
	\item
	Before we start doing anything, we'll usually want to move to the correct directory.
	We do this with \texttt{cd} (change directory).
	
	\item
	\texttt{cd} can take a relative or absolute path.
	This is when \texttt{..} to go up a directory becomes useful, and note that we can chain this or include it mid-path.
	\texttt{cd} can also take the special argument \texttt{-} to go back to the most recent working directory.
	Play around with it.
	
	\item
	You can also open the command line directly in a directory, by right-clicking in the file manager.
}

\subsection{Inspection}

\begin{frame}{Looking Around}
	\begin{itemize}
		\item \texttt{ls}: LiSt directory contents
		\begin{itemize}
			\item \texttt{ls -a} includes hidden files/directories (e.g.\ \texttt{.git}, {.gitignore})
			\item \texttt{ls -l} gives extra info
			\item \texttt{ls -al} does both
		\end{itemize}
	\end{itemize}
\end{frame}
\note[itemize]{
	\item
	When you're in a directory, you want to know which files are there.
	Or which directories, so you can \texttt{cd} further in.
	\texttt{ls} tells you.
	
	\item
	\texttt{ls} takes options, a.k.a. \emph{flags}.
	\texttt{-a} (all) shows hidden files and directories, which start with a dot.
	This is important, 'cause Git uses these.
	\texttt{-l} (long) shows owner, permissions, size, date, etc.
	
	\item
	A hyphen and a single letter is the convention for flags in the shell.
	We can combine options by giving them sequentially.
	A single hyphen followed by multiple letters also combines flags.
	Order doesn't matter.
	Demonstrate this.
	
	\item
	Some programs take long options: these use two hyphens followed by a word.
	We'll see these later on some Git commands.
}

\begin{frame}{Looking at Files}
	\begin{itemize}
		\item \texttt{cat}: print a file
		\item \texttt{head}: print the first 10 lines of a file
		\item \texttt{tail}: print the last 10 lines of a file
		\item \texttt{less}: open a file in the pager
	\end{itemize}
\end{frame}
\note[itemize]{
	\item
	There are a variety of ways we can look at the contents of a file in the command line.
	These all work within the command line, so they'll only work for text files.
	
	\item
	The names are a bit funny.
	\texttt{cat} comes from ``concatenate'', because it can print several files back-to-back.
	\texttt{less} is a better version of \texttt{more}.
}

\begin{frame}{Opening Files \& Directories}
	\begin{itemize}
		\item Open a file with the OS:
		\begin{itemize}
			\item \texttt{open} on Mac OS X
			\item \texttt{xdg-open} in Linux
			\item \texttt{start} in Git Bash on Windows
		\end{itemize}
		\item Using these on a directory opens the file manager
	\end{itemize}
\end{frame}
\note[itemize]{
	\item
	If you want to use your normal text editor, or work with non-text files, you can open files as though you'd double-clicked them in your file manager.
	How to do this differs by OS.
	
	\item
	You can also use these to open your file manager, which can be helpful to move or copy files if you're not familiar with doing so on the command line.
}

\subsection{Modification}

\begin{frame}{Changing Things}
	\begin{itemize}
		\item \texttt{mkdir}: MaKe DIRectory
		\item \texttt{rmdir}: ReMove empty DIRectory
		\item \texttt{touch}: create blank file, or change last modified date
		\item \texttt{cp}: CoPy a file
		\begin{itemize}
			\item \texttt{cp -r}: CoPy a directory (recursive)
		\end{itemize}
		\item \texttt{mv}: MoVe/rename a file/directory
		\item \texttt{rm}: ReMove a file
		\begin{itemize}
			\item \texttt{rm -i} asks confirmation: safer
			\item \texttt{rm -r} removes a directory and everything inside: DANGER!
		\end{itemize}
	\end{itemize}
\end{frame}
\note[itemize]{
	\item
	We can use all sorts of commands to create, remove, copy, or remove files or directories.
	I won't go into the details now, and you should be able to do most of this with your file manager.
	But this might be useful in future, and you might see me using these out of habit.
}

\end{document}
