\usetheme{Madrid}
\usecolortheme{seahorse}


\title{An Introduction to Git}
\author{Christopher Brown}
\date{}

\begin{document}

\begin{frame}
	\titlepage
\end{frame}
\note[itemize]{
	\item
	I'm going to teach this over several sessions.
	Hopefully the first session will teach enough to just spin up a Git repository and start making commits.
	But if that's all you ever do, you're missing out on most of the benefits that version control will give you.
	So subsequent sessions will focus on how to get the most out of Git, and use it to be more productive.
}

\part{Prerequisites}

\begin{frame}
	\partpage
\end{frame}
\note[itemize]{
	\item
	Before we start looking at Git, there're some prerequisites we need to go over.
	
	\item
	Primarily, you'll need to know the very basics of the shell.
	We'll use Git from the command line (the shell), so it'll help to be familiar with it.
	
	\item
	There're also a few other things I'll explain the details of now, to save having to go off on tangents later.
}

\section{The Shell}

\subsection{Introduction}

\begin{frame}{Which Shell?}
	\begin{itemize}
		\item \texttt{sh}: the Bourne Shell by Stephen Bourne, 1977
		\item \texttt{bash}: the Bourne Again SHell, 1989
		\item \texttt{zsh}, \texttt{ksh}, \texttt{tcsh}, \texttt{rc}\dots
	\end{itemize}
	\texttt{bash} is default on most Linux distros.
	
	Mac OS X defaults to \texttt{zsh}.
	
	Git on Windows comes with a \texttt{bash} emulator.
\end{frame}
\note[itemize]{
	\item
	You might have heard of the shell called the ``terminal'' or the ``command line''.
	It's a text-based interface to your computer.
	You can use it instead of a graphical user interface (GUI).
	
	\item
	There are GUIs for Git, but it's good to know how to use it through the command line.
	
	\item
	Different shells, \texttt{bash}, \texttt{zsh}, etc.\ are mostly based on the Bourne shell.
	There are differences, but they're all similar enough that we shouldn't run into the differences here.
	
	\item
	On Linux or Mac OS X, you should be able to search for and open the ``terminal''.
	On Windows, you're looking for ``Git Bash''.
	Open it now.
}

\end{document}
