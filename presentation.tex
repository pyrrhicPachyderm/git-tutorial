\usetheme{Madrid}
\usecolortheme{seahorse}


\usepackage{amsmath}
\usepackage{siunitx}
\usepackage{texlogos}

\title{An Introduction to Git}
\author{Christopher Brown}
\date{}

\begin{document}

\begin{frame}
	\titlepage
\end{frame}
\note[itemize]{
	\item
	I'm going to teach this over several sessions.
	Hopefully the first session will teach enough to just spin up a Git repository and start making commits.
	But if that's all you ever do, you're missing out on most of the benefits that version control will give you.
	So subsequent sessions will focus on how to get the most out of Git, and use it to be more productive.
}

\part{Prerequisites}

\begin{frame}
	\partpage
\end{frame}
\note[itemize]{
	\item
	Before we start looking at Git, there're some prerequisites we need to go over.
	
	\item
	Primarily, you'll need to know the very basics of the shell.
	We'll use Git from the command line (the shell), so it'll help to be familiar with it.
	
	\item
	There're also a few other things I'll explain the details of now, to save having to go off on tangents later.
}

\end{document}
